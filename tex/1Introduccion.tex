\section{Introducción}

\indent  La Ley del Deporte de Chile (2014) en el Artículo 1º establece que el deporte es ``aquella forma de actividad física que utiliza la motricidad humana como medio de desarrollo integral de las personas, y cualquier manifestación educativo-física, general o especial, realizada a través de la participación masiva, orientada a la integración social, al desarrollo comunitario, al cuidado o recuperación de su salud y a la recreación, como asimismo, aquella práctica de las formas de actividad deportiva o recreacional que utilizan la competición o espectáculo como su medio fundamental de expresión social, y que se organiza bajo condiciones reglamentadas, buscando los máximos estándares de rendimiento'' (Ministerio del Interior, 2001). 

\indent La actividad física forma parte central de la formación integral de toda persona. El deporte es parte esencial de una vida saludable, además de vital para mantener una mente sana, e incluso puede afectar de forma positiva en el comportamiento de las personas. En este contexto, se puede observar que en los colegios se ha estado fomentando la realización del deporte en sus alumnos, aumentando la cantidad de deportes que imparten y con ello las competencias deportivas en las que participan. 

\indent De forma de acotar el problema a analizar, se estudiará el caso del colegio Instituto Hebreo y nos centraremos en periodos de un mes. Además, se pondrá énfasis en los deportes más populares que se realizan en dicho establecimento, los que corresponden a: fútbol, básquetbol, hándbol, vóleibol y fútbol sala.